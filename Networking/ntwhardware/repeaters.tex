\section{Repeaters}
A repeater is an electronic device that receives a signal and retransmits it at a higher level or higher power, or onto the other side of an obstruction, so that the signal can cover longer distances.
In telecommunication,repeater can be ann analog device that amplifies an input signal regardless of its nature (analog or digital) or a digital device that amplifies, reshapes, retimes, or performs a combination of any of these functions on a digital input signal for retransmission.
In computer networking, because repeaters work with the actual physical signal, and do not attempt to interpret the data being transmitted, they operate on the physical layer, the first layer of the OSI model.
Repeaters are used to boost signals in coaxial and twisted pair cable and in optical fiber lines. An electrical signal in a cable gets weaker the further it travels, due to energy dissipated in conductor resistance and dielectric losses. Similarly a light signal traveling through an optical fiber suffers attenuation due to scattering and absorption. In long cable runs, repeaters are used to periodically regenerate and strengthen the signal.