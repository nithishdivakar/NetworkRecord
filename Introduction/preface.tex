\chapter*{Preface}
\thispagestyle{empty}
The book named "Network Programming Laboratory Manual� 
is designed in connection with the network programming lab in the 
Seventh Semester Computer Science and Engineering. Attempts have been made to 
present network concepts, Berkeley sockets, network programming in java, 
Java Database Connectivity(JDBC), Remote method invocation(RMI), Extensile Markup 
language(XML), Java Server pages(JSP) etc. 

In addition, examples in areas of Connection oriented communication,
connection less communication, JDBC, Remote Method Invocation, XML and Java Server Pages have been worked out
in this record. The major features of the record are
\begin{enumerate}
	\item Simple presentation
	\item Coverage of network related packages in java.
	\item Explanations of networking concepts, JDBC, RMI, XML, JSP etc.
	\item Illustrative examples
\end{enumerate}
The record is organized as three parts
\begin{itemize}
	\item In Part I, the essential concepts related to Computer network, protocols and network models are explained.
	\item Part II contains the topics of network programming in java, JDBC, RMI, XML, JSP etc.
	\item Part III layout illustrative examples of the technologies in Part II.
	\item Part IV list all reference materials  which have been used to develop this record 
\end{itemize}