\section{What is XML}
Extensible Markup Language or shortly XML is simply a set of rules used to impose structure and syntax on plain text.
XML allows hierarchical and structured data to be represented with ease.
XML is a subset of Standard General Markup Language (SGML) and is defined in XML 1.0 Specification produced by the W3C.
XML defines a set of rules for encoding documents in a format that is both human-readable and machine-readable.
It allows  creating  documents that are platform independent
This property of XML allows it to be  used to exchange data over a network. 
XML achieves it structuring using markups or tags which are placed over plain data.
The tags can be nested indefinitely allowing XML to represent highly complex and hierarchical data with ease.
Similar to HTML documents,  XML documents can be created in a text editor, such as Notepad which can be then saved in .xml extension.
Unlike HTML which have predetermined tags, XML�s tags are user defined and hence XML is highly extensible.
XML documents are processed by XML parsers and the data is passed to the applications interested in using them.