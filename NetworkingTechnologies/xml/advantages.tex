\section{Advantages of using XML}
XML because of its SGML foundation, is flexible enough to work for any situation that requires the formatting of structured data. Any information that can be stored in a spreadsheet or database type structure can be stored within an XML document.
XML document have very strict sidelines that must be followed in order for them to be read by the variety of parsing and validating software available for use with XML documents on the internet. Documents that meet the requirements are termed well-formed, having followed all of the rules for creating XML documents and applications.
XML is highly extensibility and has a universal format. So it is  widely used for data exchange between applications and networks.
XML allows  creation of custom tags  based on the requirements of a document. Therefore, they can be used as a vocabulary for all related documents. 
The simplicity offered by XML allows it to be  easy to understood and used.
The organization abilities XML allows structuring even complex data and seamless building of  platform by segmenting the design process.
Separation of data and formatting rules simplifies the development process.
XML is an international standard. This implies that an XML document can be processed by all application s with ease.
Highly extensible property of XML in form of creating DTDs and personalized tags allows it be used for all types of applications with ease. XML allows structured data transportation and handling easy.