\section{Introduction to JSP}
JavaServer Pages (JSP) is a technology that helps software developers create dynamically generated web pages based on HTML, XML, or other document types.
JSP is a server side script, but it uses the Java programming language. 
To deploy and run JavaServer Pages, a compatible web server with a servlet container, such as Apache Tomcat, is required.
A discussion about JSP would be incomplete without an overview about Java servlets because JSP can be viewed as a high-level abstraction of Java servlets.

A Servlet is a Java-based server-side web technology. Technically speaking, a Servlet is a Java class that conforms to the Java Servlet API, a protocol by which a Java class may respond to requests. Servlets could in principle communicate over any client�server protocol.
JSPs are translated into servlets at runtime.
Each JSP's servlet is cached and re-used until the original JSP is modified.

JSP allows Java code to be embedded within html code in a JSP page so that certain pre-defined actions can be interleaved with static web markup content, with the resulting page being compiled and executed on the server to deliver a document.
The compiled pages, as well as any dependent Java libraries, use Java byte-code rather than a native software format.
Like any other Java program, they must be executed within a Java virtual machine (JVM) that integrates with the server's host operating system to provide an abstract platform-neutral environment.
JSP pages are usually used to deliver HTML and XML documents, but through the use of OutputStream, they can deliver other types of data as well.





