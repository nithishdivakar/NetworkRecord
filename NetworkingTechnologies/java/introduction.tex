%Introduction
%why we prefer java for networking?
%Things offered by java to make network programming simple(.net)
%how java implements connection oriented TCP and connection less UDP protocols
%rmi and jdbc

\section{Network Programming and Java}
Java is practically a synonym for Internet Programming. There are a number of reasons for this, not the least of which is its ability to generate secure, cross-platform, portable code. However, one of the most important reasons for Java being the premier language for network programming are the classes defined in java.net package. They provide an easy-to-use means by which programmers of all skill level can access network resources.

At the core of Java`s networking support is the concept of a \emph{socket}. A socket identifies an end point in a network.