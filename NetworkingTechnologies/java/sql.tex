%(.sql package)
%	why use database? JDBC?
%	important classes for database 

\section{Java.sql Package}
JDBC is a Java-based data access technology which provides methods for querying and updating data in a database. This technology is an API for the Java programming language that defines how a client may access a database. The classes that makes up the Application Programming Interface of JDBC technology  are contained in the Java.sql package. Java.sql package constitutes an API providing a framework which allows different drivers to be installed dynamically to access different data sources. Although the JDBC API is mainly geared to passing SQL statements to a database, it provides for reading and writing data from any data source with a tabular format. The reader/writer facility, available through the javax.sql.RowSet group of interfaces, can be customized to use and update data from a spread sheet, flat file, or any other tabular data source.
\subsection{Classes of Java.sql Package}
\begin{enumerate}
	\item \textbf{DriverManager}\\
	This class provides the basic services for managing a set of JDBC drivers. As part of its initialization, the DriverManager class will attempt to load the driver classes referenced in the "jdbc.drivers" system property. This allows a user to customize the JDBC Drivers used by their applications.	The most important method of DriverManager class is the \emph{getConnection()}. It has three forms which is described below.
	\begin{enumerate}
		\item[] \emph{static Connection	getConnection(String url)throws SQLException}
         		\item[] \emph{static Connection	getConnection(String url, Properties info)throws SQLException}
		\item[] \emph{static Connection	getConnection(String url, String user, String password)throws SQLException}
	\end{enumerate}
	The \emph{getConnection()} method attempts to establish a connection to the given database URL. The DriverManager attempts to select an appropriate driver from the set of registered JDBC drivers.
\end{enumerate}

\subsection{Interfaces of Java.sql Package}
\begin{enumerate}
	\item \textbf{Connection}\\
	The Connection interface describes a connection(session) with a specific database. It allows SQL statements to be executed and results to be returned within the context of a connection. The most important method provided by this interface is the \emph{createStatement()}. It�s Syntax is described below.
	 \begin{enumerate}
		\item[] \emph{Statement createStatement()}
	\end{enumerate}
The \emph{createStatement()} creates a Statement object for sending SQL statements to the database.
	
	\item \textbf{ResultSet}\\
	The ResultSet interface represents a table of data representing a database result set, which is usually generated by executing a statement that queries the database. A ResultSet object maintains a cursor pointing to its current row of data. Initially the cursor is positioned before the first row. The \emph{next()} method moves the cursor to the next row, and because it returns false when there are no more rows in the ResultSet object. The ResultSet interface provides \emph{getter} methods (getBoolean, getLong, and getString) for retrieving column values from the current row. Values can be retrieved using either the index number(\emph{integer value}) of the column or the name of the column(\emph{String}).
	 
	\item \textbf{Statement}\\
	The Statement interface describes an object used for executing a static SQL statement and returning the results it produces. The most important methods of Statement interface are.
	 \begin{enumerate}
		\item[] \emph{ResultSet executeQuery(String sql)throws SQLException}
		\item[] \emph{int executeUpdate(String sql)throws SQLException} 
	\end{enumerate}
	The \emph{executeQuery()}  methods executes the given SQL statement, which returns a single ResultSet object while \emph{executeUpdate()} method Executes the given SQL statement, which may be an INSERT, UPDATE, or DELETE statement or an SQL statement that returns nothing, such as an SQL DDL statement.
\end{enumerate}
	  