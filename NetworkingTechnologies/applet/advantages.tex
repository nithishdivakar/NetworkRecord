\section{Advantages}
A Java applet has the following advantages:
\begin{itemize}
\item It is simple to make it work on Linux, Microsoft Windows and OS X i.e. to make it cross platform. Applets are supported by most web browsers.
\item The same applet can work on "all" installed versions of Java at the same time, rather than just the latest plug-in version only.
\item Most web browsers cache applets so will be quick to load when returning to a web page. Applets also improve with use: after a first applet is run, the JVM is already running and starts quickly (the JVM will need to restart each time the browser starts afresh).
\item It can move the work from the server to the client, making a web solution more scalable with the number of users/clients.
\item If a standalone program (like Google Earth) talks to a web server, that server normally needs to support all prior versions for users which have not kept their client software updated. In contrast, a properly configured browser loads (and caches) the latest applet version, so there is no need to support legacy versions.
\item The applet naturally supports the changing user state, such as figure positions on the chessboard.
\item Developers can develop and debug an applet direct simply by creating a main routine (either in the applet's class or in a separate class) and calling init() and start() on the applet, thus allowing for development in their favorite Java SE development environment. All one has to do after that is re-test the applet in the 
\item AppletViewer program or a web browser to ensure it conforms to security restrictions.
\item An untrusted applet has no access to the local machine and can only access the server it came from. This makes such an applet much safer to run than a standalone executable that it could replace. However, a signed applet can have full access to the machine it is running on if the user agrees.
\item Java applets are fast - and can even have similar performance to native installed software.
\end{itemize}