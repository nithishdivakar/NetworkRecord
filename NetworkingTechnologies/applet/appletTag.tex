\section{The HTML APPLET Tag}
The APPLET tag is used to start an applet from both an HTML document and from an applet viewer.
Web browsers allow many applets on a single page.
The syntax for the standard APPLET tag is shown here. Square bracketed items are optional.\\
\begin{tabular}{l l l}
&$< APPLET$\\
&$\;\;\;$[CODEBASE = codebaseURL] \\
&$\;\;\;$CODE = appletFile [ALT = alternateText] \\
&$\;\;\;$[NAME = appletInstanceName] \\
&$\;\;\;$WIDTH = pixels HEIGHT = pixels\\
&$\;\;\;$[ALIGN = alignment] [VSPACE = pixels] \\
&$\;\;\;$[HSPACE = pixels]\\
& $>$\\
&$\;\;\;$[$<$ PARAM NAME = AttributeName VALUE = AttributeValue$>$] \\
&$\;\;\;$[$<$ PARAM NAME = AttributeName2 VALUE = AttributeValue$>$]\\
&$\;\;\;$...\\
&$\;\;\;$[HTML Displayed in the absence of Java] \\
&$<$/APPLET$>$\\
\end{tabular}
\\\\
\textbf{CODEBASE}	optional attribute that specifies the base URL of the applet code.\\
\textbf{CODE} CODE is a required attribute that gives the name of the file containing your applet�s compiled .class file. \\
\textbf{ALT} The ALT tag is an optional attribute used to specify a short text message that should be displayed if the browser understands the APPLET tag but can�t currently run Java applets. \\
\textbf{NAME}	NAME is an optional attribute used to specify a name for the applet instance. Applets must be named in order for other applets on the same page to find them by name and communicate with them. \\
\textbf{ALIGN}	ALIGN is an optional attribute that specifies the alignment of the applet. \\
\textbf{VSPACE AND HSPACE} These attributes are optional. VSPACE specifies the space, in pixels, above and below the applet. HSPACE specifies the space, in pixels, on each side of the applet.\\
\textbf{PARAM NAME AND VALUE}	The PARAM tag allows you to specify applet- specific arguments in an HTML page. Applets access their attributes with the getParameter( ) method.