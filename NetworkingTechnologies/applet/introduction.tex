\section{Java Applet}
Applets are small applications that are accessed on an Internet server, transported over the Internet, automatically installed, and run as part of a Web document. Applet is a client side technology as far  as networking is concerned. After an applet arrives on the client, it has limited access to resources, so that it can produce an arbitrary multimedia user interface and run complex computations without introducing the risk of viruses or breaching data integrity.

Java Applets can provide web applications with interactive features that cannot be provided by HTML. Since Java's bytecode is platform-independent, Java applets can be executed by browsers running under many platforms, including Windows, Unix, Mac OS, and Linux. When a Java technology-enabled web browser processes a page that contains an applet, the applet's code is transferred to the client's system and executed by the browser's Java Virtual Machine (JVM). An HTML page references an applet either via the deprecated $<APPLET>$
