\program{Berkeley Socket}{}
\subsection*{Description}
Here, in this experiment, we have created a client  program which communicate with server program.
The server reverses the string sent by client and sends it back.
\subsection*{Algorithm}
\algorithm{BerkeleyServer}{}{
\textbf{Step 1:} Start Server\\
\textbf{Step 2:} Wait for a connect request from the client\\
\textbf{Step 3:} Accept the connection request\\
\textbf{Step 4:} Read a \emph{message} from client\\
\textbf{Step 5:} Find reverse of \emph{message} and store it on \emph{reverse\_mesage}\\
\textbf{Step 6:} Send the \emph{reverse\_message} back to client\\
\textbf{Step 7:} Goto Step 2
}\\
\algorithm{BerkeleyClient}{}{
\textbf{Step 1:} Start Client\\
\textbf{Step 2:} Connect to server\\
\textbf{Step 3:} Read a \emph{message} from user\\
\textbf{Step 4:} Send the \emph{message} to the server\\
\textbf{Step 5:} Read a \emph{message} from server\\
\textbf{Step 6:} Print the \emph{message}\\
\textbf{Step 7:} Stop
}

\subsection*{Program}
\subsubsection{berkeleyServer.c}
\includecode{./Experiments/COMMON/berkeley/berkeleyServer.java}
\subsubsection{berkeleyClient.c}
\includecode{./Experiments/COMMON/berkeley/berkeleyClient.java}


\outputscreen{./Experiments/COMMON/berkeley/berkeleyClient.png}
\outputscreen{./Experiments/COMMON/berkeley/berkeleyServer.png}

\subsection{Result}\result
