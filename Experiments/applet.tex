\program{Java Applet}{02-12-2012}
\subsection{Description}
A Java applet is an applet delivered to users in the form of Java byte-code. Java applets can run in a Web browser using a Java Virtual Machine (JVM), or in Sun's AppletViewer, a stand-alone tool for testing applets.
Java applets run at very fast speeds than JavaScript.
In addition they can use 3D hardware acceleration that is available from Java.
This makes applets well suited for non-trivial, computation intensive visualizations to  web based applications entirely computed in the client side.
In this Chapter we describe creation of an applet which can draw a circle and then daw a continuously revolving tangent to the circle.
This result can be displayed in a web browser easily.


\subsection{Algorithm}
\algorithm{TangentToCircle($r$,$x_c$,$y_c$,L)}{}{
//$r$ is the radius of the circle to be drawn with center at point ($x_c$,$y_c$)\\
//The length of the tangent is L\\
//The end points of tangent is derived from the fact that the locus of end points of the\\
//tangent revolving around the circle forms a bigger circle with radius $\sqrt{L^2+r^2}$ and the\\
//angular separation between the 2 points is $tan^{-1}(\frac{L}{r})$\\
//\emph{drawCircle($r$,$x_c$,$y_c$)} draws a circle centered at ($x_c$,$y_c$) with radius $r$\\
//\emph{drawLine($x_1$,$y_1$,$x_2$,$y_2$)} draws a line from ($x_1$,$y_1$) to ($x_2$,$y_2$)\\
\textbf{Step 1:} Start the program\\
\textbf{Step 2:} Set  $\theta$:=0 \\
\textbf{Step 3:} $\theta$:=$\theta$+1\\
\textbf{Step 4:} $x_1$:=$\sqrt{L^2+r^2}*cos(\theta)$\\
\textbf{Step 5:} $y_1$:=$\sqrt{L^2+r^2}*sin(\theta)$\\
\textbf{Step 6:} $x_2$:=$\sqrt{L^2+r^2}*cos(\theta+tan^{-1}(\frac{L}{r}))$\\
\textbf{Step 7:} $y_2$:=$\sqrt{L^2+r^2}*sin(\theta+tan^{-1}(\frac{L}{r}))$\\
\textbf{Step 8:} \emph{drawCircle($r$,$x_c$,$y_c$)}\\
\textbf{Step 9:} \emph{drawLine($x_1$,$y_1$,$x_2$,$y_2$)}\\
\textbf{Step 10:} Goto Step 3\\
\textbf{Step 11:} Stop\\
}

\subsection{Program}
\includecode{./Experiments/code/applet.java}

\outputscreen{./Experiments/output/applet.jpg}

\subsection{Result}\result
