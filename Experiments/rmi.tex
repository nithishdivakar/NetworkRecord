\program{Remote Method Invocation}{19-10-2012}
\subsection{Description}
A remote procedure call (RPC) is an inter-process communication that allows a computer program to cause a subroutine or procedure to execute in another address space (commonly on another computer on a shared network) without the programmer explicitly coding the details for this remote interaction. That is, the programmer writes essentially the same code whether the subroutine is local to the executing program, or remote. When the software in question uses object-oriented principles, RPC is called remote invocation or remote method invocation.RMI is the Java version of  a remote procedure call (RPC), but with the ability to pass one or more objects along with the request. 

An RPC is initiated by the client, which sends a request message to a known remote server to execute a specified procedure with supplied parameters. The remote server sends a response to the client, and the application continues its process. While the server is processing the call, the client is blocked (it waits until the server has finished processing before resuming execution).

Sequence of events during a RPC
\begin{enumerate}
\item The client calls the client stub. The call is a local procedure call, with parameters pushed on to the stack in the normal way.
\item The client stub packs the parameters into a message and makes a system call to send the message. Packing the parameters is called marshaling.
\item The client's local operating system sends the message from the client machine to the server machine.
\item The local operating system on the server machine passes the incoming packets to the server stub.
\item The server stub unpacks the parameters from the message . Unpacking the parameters is called unmarshmaling.
\item Finally, the server stub calls the server procedure. The reply traces the same steps in the reverse direction.
\end{enumerate}

\subsection{Algorithm}
\algorithm{Remote Method Invocation}{}{
// The four source files used are:\\
//\textbf{AddServerIntf.java} defines remote interface that is provided by the server. \\
//\textbf{AddServerImpl.java} implements the methods in remote interface.\\
//\textbf{AddServer.java} contains main program for server machine. its primary function is to\\
//update the RMI registry on that machine.\\
//\textbf{AddClient.java} implements the client side of the distributed application.\\
\\
\textbf{Step 1} :Create and Compile the source codes\\
This will create files \textbf{AddClient.class, AddServerIntf.class, AddServerImpl.class, AddServer.class}\\
\textbf{Step 2} :Generate a Stub using RMI compiler.\\
This will create file \textbf{AddServerImpl\_Stub.class}\\
\textbf{Step 3} :Install files on client and server machines\\
Copy \textbf{AddClient.class}, \textbf{AddServerImpl\_Stub.class}, and \textbf{AddServerIntf.class} to a directory on the client machine.\\
Copy \textbf{AddServerIntf.class}, \textbf{AddServerImpl.class}, \textbf{AddServerImpl\_Stub.class}, and \textbf{AddServer.class} to a directory on the server machine.\\
\textbf{Step 4} :Start RMI registry on the Server Machine\\
\textbf{Step 5} :Start the Server\\
\textbf{Step 6} :Start the Client
}

\subsection{Program}
\includecode{./Experiments/code/rmi.java}

\outputscreen{./Experiments/output/rmi.jpg}

\subsection{Result}\result
