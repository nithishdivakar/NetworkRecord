\program{Broadcasting}{17-10-2012}
\subsection{Description}
Broadcasting refers to a method of transferring a message to all recipients simultaneously. In computer networking, broadcasting refers to transmitting a packet that will be received by every device on the network.
In this chapter, we describe an implementation of a server which broadcasts a specific message to all the clients connected to it.
The server is capable of handling multiple clients at a time.
A simple implementation of the client is also described.



\subsection{Algorithm}
\algorithm{BroadcastServer}{}{
\textbf{Step 1:} Start the server\\
\textbf{Step 2:} Read a line from user\\
\textbf{Step 3:} Send this line to all connected clients\\
\textbf{Step 4:} Goto Step 2
}\\
\algorithm{BroadcastClient}{}{
\textbf{Step 1:} Connect to the server\\
\textbf{Step 2:} Receive message coming from server\\
\textbf{Step 3:} Print the message\\
\textbf{Step 4:} Goto Step 2
}

\subsection{Program}
\includecode{./Experiments/code/broadcast.java}


\outputscreen{./Experiments/output/broadcast.jpg}

\subsection{Result}\result
