\program{Implementation of Ping Command}{07-12-2012}
\subsection{Description}
The ping command is a Command Prompt command available in linux and windows and many other operating system.
This command is used to test the ability of the source computer to reach a specified destination computer. 
The ping command is usually used as a simple way verify that a computer can communicate over the network with another computer or network device.
Ping command works by sending echo requests to destination computer and seeing if destination computer responds back or not.
The Standard ping command supplies much more information like round trip time.
In this experiment we have tried to implement an limited ping command which can test wether a destination computer can be reached or not.

\subsection{Algorithm}
\algorithm{Ping}{}{
\textbf{Step 1:}Start the program\\
\textbf{Step 2:}Start timer\\
\textbf{Step 3:}Send an echo request to source computer\\
\textbf{Step 4:}Wait till source computer responds\\
\textbf{Step 5:}Stop the timer when Echo request is acknowledged\\
\textbf{Step 6:}Print the value of timer\\
\textbf{Step 7:}Stop
}

\subsection{Program}
\includecode{./Experiments/COMMON/ping/ping.java}

\outputscreen{./Experiments/COMMON/ping/ping.jpg}

\subsection{Result}\result
